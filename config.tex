\documentclass[%
  paper=a4,
  fontsize=10pt,
  ngerman
  ]{scrartcl}

% Basics für Codierung und Sprache
% ===========================================================
  \usepackage{shellesc}                 % Compiler-Option -shell-escape benutzen!
  \usepackage[final]{graphicx}          % Einbindung von Grafiken
  \usepackage{subcaption}
%  \usepackage[utf8]{inputenc}          % Dateien sind UTF8-codiert
  \usepackage{babel}                    % deutsche Silbentrennung, etc.
  \usepackage[german=quotes]{csquotes}  % deutsche Anführungszeichen mit \enquote{...}
% ===========================================================

% Fonts und Typographie
% ===========================================================
  \usepackage[babel=true,final,tracking=smallcaps]{microtype}
  \DisableLigatures{encoding = T1, family = tt* }   % keine Ligaturen für Monospace-Fonts
% ===========================================================

% Farben
% ===========================================================
  \usepackage[usenames,x11names,final]{xcolor}
% ===========================================================

% Mathe-Pakete und -Einstellungen
% ===========================================================
  \usepackage{mathtools}               % Tools zum Setzen von Formeln
  \usepackage{amssymb}                 % übliche Mathe-Symbole
  \usepackage[bigdelims]{newtxmath}    % moderne Mathe-Font
  \allowdisplaybreaks                  % seitenübergreifende Rechnungen
  \usepackage{bm}                      % math bold font
  \usepackage{wasysym}                 % noch mehr Symbole
% ===========================================================

% TikZ
% ===========================================================
  \usepackage{tikz}
  \usetikzlibrary{arrows,arrows.meta}    % mehr Pfeile!
  \usetikzlibrary{calc}                  % TikZ kann rechnen
  \usetikzlibrary{positioning}
  \tikzset{>=Latex}                      % Standard-Pfeilspitze
% ===========================================================

% Seitenlayout, Kopf-/Fußzeile
% ===========================================================
  \usepackage{scrlayer-scrpage}
  \pagestyle{scrheadings}
  \usepackage[top=5cm, bottom=3cm, left=2.5cm, right=2cm]{geometry}
  \clearscrheadfoot 
  \setheadsepline{0.4pt}                            % Linie in Kopfzeile
  \setfootsepline{0.4pt}                            % Linie in Fußzeile
  \setkomafont{section}{\fontsize{14bp}{18.8bp}\normalfont}  % Schriftart der Section
  \setkomafont{subsection}{\fontsize{12bp}{16bp}\normalfont}                  
  \setkomafont{pagehead}{\textnormal}                 % Schriftart Kopfzeile
  \setkomafont{pagefoot}{\normalfont\footnotesize}  % Schriftart Fußzeile 
  \cfoot{\thepage}                                  % Seitenzahl unten Mitte
  \lohead{\obenlinks}                               % Titel oben links
  \raggedbottom                                     % Flattersatz
  \usepackage{setspace}                             % erweiterte Abstandsoptionen
  \onehalfspacing                                   % Zeilenabstand 1.5-fach
  \setlength{\parindent}{0pt}                       % Einrückung neuer Absätze
  \setlength{\parskip}{0.5\baselineskip}            % Abstand neuer Absätze
% ===========================================================

% Hyperref für Referenzen und Hyperlinks
% ===========================================================
  \usepackage[%
    hidelinks,
    pdfpagelabels,
    bookmarksopen=true,
    bookmarksnumbered=true,
    linkcolor=black,
    urlcolor=SkyBlue2,
    plainpages=false,
    pagebackref,
    citecolor=black,
    hypertexnames=true,
    pdfborderstyle={/S/U},
    linkbordercolor=SkyBlue2,
    colorlinks=false,
    backref=false]{hyperref}
  \hypersetup{final}
% ===========================================================

% Listen und Tabellen
% ===========================================================
  \usepackage{multicol}
  \usepackage[shortlabels]{enumitem}
  \setlist{itemsep=0pt}
  \setlist[enumerate]{font=\sffamily\bfseries}
  \setlist[itemize]{label=$\triangleright$}
  \usepackage{tabularx}
% ===========================================================

% minted
% ===========================================================
\usepackage{minted}
\setminted{%
  style=default,
  fontsize=\small,
  breaklines,
  breakanywhere=false,
  breakbytoken=false,
  breakbytokenanywhere=false,
  breakafter={.,},
  autogobble,
  numbersep=3mm,
  tabsize=4,
  linenos,
  frame=lines
}
\setmintedinline{%
  fontsize=\normalsize,
  numbers=none,
  numbersep=12pt,
  tabsize=4,
}

%%%%%%%%%%%%%%%%%%%%%%%%%%%%%%%%%%%%%%%%%%%%%%%%%%%%%%%%%%%
%%% Ab hier folgen nur noch vordefinierte Mathe-Befehle %%%
%%%%%%%%%%%%%%%%%%%%%%%%%%%%%%%%%%%%%%%%%%%%%%%%%%%%%%%%%%%

\newcommand{\BB}{\mathbb{B}}
\newcommand{\CC}{\mathbb{C}}
\newcommand{\NN}{\mathbb{N}}
\newcommand{\QQ}{\mathbb{Q}}
\newcommand{\RR}{\mathbb{R}}
\newcommand{\ZZ}{\mathbb{Z}}
\newcommand{\oh}{\mathcal{O}}            
\newcommand{\ol}[1]{\overline{#1}}
\newcommand{\wt}[1]{\widetilde{#1}}
\newcommand{\wh}[1]{\widehat{#1}}

\DeclareMathOperator{\id}{id}                        % Identität
\DeclareMathOperator{\pot}{\mathcal{P}}              % Potenzmenge

% Klammerungen und ähnliches
\DeclarePairedDelimiter{\absolut}{\lvert}{\rvert}    % Betrag
\DeclarePairedDelimiter{\ceiling}{\lceil}{\rceil}    % aufrunden
\DeclarePairedDelimiter{\Floor}{\lfloor}{\rfloor}    % aufrunden
\DeclarePairedDelimiter{\Norm}{\lVert}{\rVert}       % Norm
\DeclarePairedDelimiter{\sprod}{\langle}{\rangle}    % spitze Klammern
\DeclarePairedDelimiter{\enbrace}{(}{)}              % runde Klammern
\DeclarePairedDelimiter{\benbrace}{\lbrack}{\rbrack} % eckige Klammern
\DeclarePairedDelimiter{\penbrace}{\{}{\}}           % geschweifte Klammern
\newcommand{\Underbrace}[2]{{\underbrace{#1}_{#2}}}  % bessere Unterklammerungen
% Kurzschreibweisen für Faule und Code-Vervollständigung
\newcommand{\abs}[1]{\absolut*{#1}}
\newcommand{\ceil}[1]{\ceiling*{#1}}
\newcommand{\flo}[1]{\Floor*{#1}}
\newcommand{\no}[1]{\Norm*{#1}}
\newcommand{\sk}[1]{\sprod*{#1}}
\newcommand{\enb}[1]{\enbrace*{#1}}
\newcommand{\penb}[1]{\penbrace*{#1}}
\newcommand{\benb}[1]{\benbrace*{#1}}
\newcommand{\stack}[2]{\makebox[1cm][c]{$\stackrel{#1}{#2}$}}